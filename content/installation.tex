\subsection{To download}
\frame{
	\frametitle{You need the following}
	
	\begin{itemize}
		\item \href{https://netbeans.org/downloads/}{NetBeans 8.1}
		\item<2-> \href{http://www.oracle.com/technetwork/java/javafxscenebuilder-1x-archive-2199384.html\#javafx-scenebuilder-2.0-oth-JPR}{JavaFX Scene Builder 2.0}
		\item<3-> \href{http://www.oracle.com/technetwork/java/javase/downloads/jdk8-downloads-2133151.html}{JDK 1.8}
		\item<4-> \href{https://github.com/JodaOrg/joda-time/releases/download/v2.9/joda-time-2.9-dist.tar.gz}{Joda-Time v 2.9}
	\end{itemize}
}

\subsection{Configuration}
\frame{
	\frametitle{Integrate Scene Builder 2.0 into NetBeans}
	
	\begin{block}{Windows}
		Tools/Options/Java/JavaFX/\\
		
	\end{block}
	
	\begin{block}{Mac OS X}
		NetBeans/Preferences/Java/JavaFX/
	\end{block}
	
	\begin{block}{Linux}
		NetBeans/Tools/Options/Java/JavaFX/
	\end{block}
}

\subsection{Project}
\frame{
	\frametitle{Create the Project}
	
	Create a new project in NetBeans.
	
	\begin{block}{Project informations}
		Projecttyp: Java FX FXML Application\\
		Name: JavaFXClock\\
		FXML name: Clock
	\end{block}
	
	\begin{block}{Package structure}
		javafxclock/controller\\
		javafxclock/model\\
		javafxclock/style\\
		javafxclock/util\\
		javafxclock/view
	\end{block}
}
