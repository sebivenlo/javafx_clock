
\begin{frame}{Task 1}
\begin{block}{First steps}
Drag four labels on the pallette in SceneBuilder. Label the label like  this: “Day”, “HH”, “mm”, “SS”, 
\end{block}
\pause
\begin{exampleblock}{Hint}
You can change the text of labels by clicking on each seperate label and changing the “text” value.  
\end{exampleblock}
\end{frame}
\begin{frame}{Task 2}
\begin{block}{Label properties}
Now we want different sizes for each of the 3 time values. This to indicate that hours are bigger than minutes, which are bigger than seconds. 
\end{block}
\pause
\begin{exampleblock}{Hint}
Making the textsize of a label bigger can be done using the scenebuilder, or modifying the textfile of the view. In the scenebuilder we select the hour label and change the fontsize to 36px in the font drop menu.
\end{exampleblock}
\end{frame}
\begin{frame}{Task 3}
\begin{block}{styling}
Style your clock with css files. Add a CSS file to your project and to your scene.
\end{block}
\pause
\begin{exampleblock}{Hint}
There are differet ways to style a FXML file. In the SceneBuilder you can

\end{exampleblock}
\end{frame}
\begin{frame}{Tasks 4}
\begin{block}{Buttons}
Add a button to the screen which enables user interaction with the 
application. Use simple system.out.printline to show that the button has been clicked. 
\end{block}
\pause
\begin{exampleblock}{Hint}

\end{exampleblock}
\end{frame}